\documentclass{beamer}
%\usepackage[latin1]
%\usepackage[utf8x]{inputenc}
%\usepackage[cm-default]{fontspec}
%\usepackage{graphicx}
%\usepackage{subfigure}
%\usepackage{wrapfig}
%\usepackage{xunicode}
%\usepackage{xltxtra}
%\usepackage{soul}

\usetheme{Warsaw}
\title[Ethical Hacking Seminars]{How to break into wireless LAN's enrypted with WEP/WPA}
\author{Giorgos (fedjo) Marinellis}
\institute{Foss@Ntua}
\date{April 09 2012}
\begin{document}

\begin{frame}
\titlepage
\end{frame}


\begin{frame}{Introduction}
\begin{itemize}
 \item \textbf{What's a wireless AP encryption?}
 \item A method to encrypt our network traffic and authenticate with APs and don't allow curious people see our packets
 \item \textbf{How is been achived?}
\item In general ways the packet is encrypted with a secret key. Only users who have this key can decrypt
packets.Other will see crap...
\end{itemize}

\end{frame}

\begin{frame}{The old way of encryption - WEP 'encrypted' w-networks }
 \begin{itemize}
  \item It's a very old encryption method based on cipher RC4 and CRC-32
\item It is provided in to methods 64bit-WEP and 128bit-WEP (and 256-WEP) and they are recognized from a 40 bit 
or 104 bit key. These strings are concatenated with a 24bit Initialization Vector(IV) to form the
RC4 key.
 \end{itemize}

\end{frame}

\begin{frame}{Authentication}
 \begin{itemize}
  \item Two methods of authentication.Open System  and Shared Key authentication
\item In Open System auth. no auth. with the AP is done.Key is only for encrypting data frames
\item In Shared Key auth. we have a 4-way handsake with the AP and then packet encryption
\item The for steps are shown below
 \end{itemize}

\end{frame}

\begin{frame}{4-way authentication with AP}
 \begin{figure}
  \includegraphics[scale=0.5]{wep_fig1}
 \end{figure}
\begin{itemize}
\item Although it is not safer to use Shared Key auth. because keystream can be derived from the challenge packets
\end{itemize}
\end{frame}

\begin{frame}{Protocol leakage...}
 \begin{figure}
  \includegraphics[scale=0.35]{wep_fig2.png}
\end{figure}
\begin{itemize}
  \item Because RC4 is a stream cipher IV must not be repetead but for an 24bit IV there is a 50\% probability
  the IV will repeat every 5000 packets.
  \item And this is where the party starts. Grabbing as many IVs from packets we can then crack the WEP key
\end{itemize}

\end{frame}

\begin{frame}{Let's get started...}
\begin{itemize}
 \item \textbf{What we will need:}
\pause \item A wireless NIC with monitor mode
\pause \item A packet sniffer
\pause \item Aircrack-ng
\end{itemize}
\end{frame}

\begin{frame}{Enable monitor mode}
 \begin{itemize}
\pause  \item \textbf{\#: airmon-ng wlan0 start}\hspace{10pt} as root\\(maybe a new virtual interface will appear)
 \end{itemize}
\end{frame}

\begin{frame}{Explore wireless networks and grab IV's}
 \begin{itemize}
\pause  \item \textbf{\$: airodump mon0 }\hspace{10pt}\hspace{10pt} to see networks 
\pause  \item \textbf{\$: airodump mon0 $<output>$  $<channel> $ 1}\hspace{10pt} to start grab IV's
 \end{itemize}
\end{frame}

\begin{frame}{The procedure...}
 \begin{itemize}
\item As we can see airodump gives as a lot info about the network
\item We can to collect at least 100.000 packets (under \# Data) 
\begin{figure}
 \includegraphics[scale=0.35]{airodumo}
\end{figure}
 \end{itemize}
\end{frame}

\begin{frame}{Cracking procedure}
 \begin{itemize}
\pause  \item \textbf{\$: aircrack-ng -a 1 -b $<bssid>$ -n $<key-length>$ $<output.ivs>$ }
\item There are several options you can use 
 \end{itemize}
\end{frame}

\begin{frame}{Traffic problems}
 \begin{itemize}
\pause \item Because we may suffer from packet un\-traffic WE have to generate more traffic 
 \pause \item That's why we use \textbf{aireplay}
 \end{itemize}
\end{frame}

\begin{frame}{ARP Injection}
 \begin{itemize} 
\pause  \item \textbf{\$: aireplay -3 -b $<AP MAC>$ -h $<Client MAC>$  mon0}
 \end{itemize}
\end{frame}

\begin{frame}{Re-send all data attack}
 \begin{itemize} 
\pause \item Ask AP to resend all packets. Some AP's re-encrypt them some use the same IV's
\pause  \item \textbf{\$: aireplay -2 -b $<AP MAC>$ -h $<Client MAC>$ -n 100 -p 0841 -c FF:FF:FF:FF:FF:FF mon0}
 \end{itemize}
\end{frame}

\begin{frame}{Fake Authentication}
 \begin{itemize} 
\pause \item Won't generate more traffic but it is uselful if there are no connected clients and we need to 
apply latter attacks
\pause \item It's easier if we have another station otherwise we must spoof ou MAC
\pause  \item \textbf{\$: aireplay -1 30 -e $<ESSID>$ -b $<BSSID>$ -h $<New MAC>$  mon0}
 \end{itemize}
\end{frame}

\begin{frame}{The new trend WPA/WPA2-PSK }
 \begin{itemize}
 \pause \item We'll talk about the PSK(Pre Shared Key) Personal edition
\pause \item There is a 2-way handsake authentication with the AP based on a secret key that each client must
know
 \end{itemize}
\end{frame}

\begin{frame}{Let's go deep}
 \begin{itemize}
 \pause \item The AP generates PMK(Pairwise Master Key) from PSK and ESSID and ESSID length hashed 4096 times with SHA-1
\pause \item Each time a client going to associate with the AP generates it's PMK
 \end{itemize}
\end{frame}

\begin{frame}{The handshake }
 \begin{itemize}
 \pause \item The AP sends to the client a random number called ANonce
\pause \item The client also generate it's random number called SNonce and mixes the PMK, ANonce, SNonce, MAC\_AP,
MAC\_Client and generates a 512 byte number called PTK 
\pause \item Then encrypts SNonce with PTK(MIC digital sign) and sends both SNonce and MIC to AP
\pause \item If the AP can match this 2 numbers then we have authentication
 \end{itemize}
\end{frame}

\begin{frame}{The new trend WPA/WPA2-PSK }
 \begin{figure}
  \includegraphics[scale=0.35]{wpa1}
 \end{figure}
\end{frame}

\begin{frame}{Where is the problem...? }
 \begin{itemize}
 \pause \item We can sniff during authentication SNonce, ANonce, MIC
\pause \item Then with bruteforcing we can use all available passphrases to check...!
\pause \item But this is an extremely long procedure
 \end{itemize}
\end{frame}

\begin{frame}{For instance}
 \begin{itemize}
 \pause \item We can use rainbow tables with precomputed PMK's for each different ESSID
\pause \item This is also a very heavy procedure but...
\pause \item we can base on peoples awareness not changing the default SSID and PSK
\pause \item There are rainbow tables free on the internet which you can use
 \end{itemize}
\end{frame}



\end{document}